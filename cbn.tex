\section{Call-by-need}

Call-by-need semantics ensures that arguments are only evaluated when needed,
and the result are memoized so that they are only evaluated once. This is in
contrast to the more common call-by-value, which always evaluates arguments, and
call-by-name, which evaluates arguments for every corresponding variable
dereference. Each semantics has advantages and drawbacks, and increasingly
programming languages are incorporating the ability to switch between them
\cite{?}. Call-by-need is used heavily to implement existing non-strict
languages such as Haskell \cite{stg}. While Haskell doesn't require call-by-need
semantics, programmers often rely on the compiler implementing memoization for
their code to have reasonable runtimes. This reliance is one of our primary
motivations to reason formally about call-by-need.  

In selecting a call-by-need semantics for our source semantics, there a couple
of well known semantics that would make for an obvious choice. In this section
we discuss why we opted to not 

One well known call-by-need semantics is Ariola et al.'s call-by-need lambda
calculus \cite{?}. It is an untyped language with the standard lambda calculus
for it's syntax: 

  $$ t := x \; | \; \lambda x . t \; | \; t \; t $$

Ariola et al. give a few different versions of their semantics. To ease 
proof burdens, we use the operational semantics, though formalizing the
syntactic account and it's connection to the operational semantics would be
interesting future work. The semantics are given in Figure~\ref{cbn}. 

Unfortunately the operational semantics have a known issue, which was
independently exposed during our formalization. The issue is that upon
evaluation of a dereferenced variable, part of the heap is forgotten. When
adding fresh variables, the semantics requires that they are fresh with respect
only to the \emph{current} heap, but when merging back with the previously
forgotten heap in the Id rule, the semantics can break the invariant that every
address in the domain of the heap is unique, and can therefore perform incorrect
evaluation. For an example of how it can go wrong see Figure~\ref{cbnwrong}, 

Another possi
\begin{figure}
\lstinputlisting{cbnbroken.v}
\caption{Original, broken call-by-need semantics}
\label{fig:cbnbroken}
\end{figure}

\begin{figure}
\lstinputlisting{cbnfixed.v}
\caption{Fixed call-by-need semantics}
\label{fig:cbnfixed}
\end{figure}

For simplicity of formalization, we choose to follow the convention of Ariola et
al. that when we bind a new variable to a new term in the heap, we ensure that
that variable is fresh respect to existing heap, without any consideration for
what subset of the heap is reachable. This prevents reasoning directly about
garbage collection or re-use of heap locations. We will return to this issue in
the discussion section, but it is worth keeping in mind throughout the paper
that relaxing the freshness constraint to being fresh with respect to
\emph{live} or \emph{reachable} heap locations should be possible, and is an
interesting opportunity for future work. 

\subsection{Call-by-name}
An important property of call-by-need is that it is an optimized implementation of
the simpler call-by-name semantics: 

\begin{align}
\inference{l \downarrow \lambda x. b \quad \quad b [m / x] \downarrow v}{l \; m \downarrow v}
\end{align}

Formally, this means that any expression that evaluates to a value in
call-by-name will evaluate to an equivalent value in call-by-need. This is a
valuable property to have, because it means we can reason about
\emph{correctness} properties of a source language, e.g. type preservation,
using these simpler evaluation semantics. 

Instead of using the above definition of call-by-name, we choose to reason
instead about curien's call by name calculus using closures. We leave proving
formally that this corresponds exactly to the above definition of call-by-name
for future work.

It's worth noting here that there are cases when requiring call-by-need
semantics is detrimental to performance. The basic idea is that there are cases
when storing a value and referencing it later is more expensive than re-creating
it on demand. For example, the function \texttt{mean xs = sum xs / length xs}
when called on a something like \texttt{[1..10000]} will memoize the list in
memory for the call to \texttt{sum}, then traverse it to compute the
\texttt{length}.  If the memoization restriction was lifted to permit any
non-strict semantics, like call-by-name, this operation could be computed more
quickly entirely in machine registers. While we choose to verify compilation of
simpl call-by-need semantics, we'll return to this discussion of relaxed
non-strict semantics in the Discussion section.

