\section{Introduction} \label{sec:intro}

Compilers are an attractive target for verification because the amortized return
on investment is high. Every time a program compiled with a verified compiler is
run, the proof ensures that the semantics of the source language are being
preserved through execution. 

Existing work has focused on verifying the correctness of \emph{strict}
languages~\cite{chlipala2007certified, leroy2012compcert}, which pre-compute
function arguments to values. This paper presents the first machine-verified
compiler that preserves correctness for a lazy language. One of the challenges
in proving that the memoization of results is implemented correctly. That is, 
variables bound at the same call site will only ever evaluate closure argument
at most once. This is a crucial optimization for non-strict languages. To  

By verifying that the is correct, we gain powerful tools for reasoning
about space and time requirements; something we could not do with only a
verification of results. This enables a large set of interesting future work,
from verifying that optimizations are indeed optimizations, to verifying that a
program \emph{really} can't go wrong, i.e. that it can't run out of space or
time. 

Our approach is enabled by a recently developed abstract machine, the Cactus
Environment Machine ($\mathcal{CE}$)~\cite{?}. $\mathcal{CE}$ uses a shared
environment to share results between instances of a variable. It can be compiled
to machine code very succinctly, reducing the load for formal reasoning about
the compiler greatly. It is likely we would not have succeeded in (or even
attempted) creating a verified compiler of call-by-need without this approach.

\subsection{Main Result}
Here we give a high level overview of the main results of the paper, along with
informal statements of the main theorems.

Our source language is lambda calculus with deBruijn indices: 
$$ t ::= i \; | \; \lambda t \; | \; t \; t $$

Application is left associative, as usual, and variables are natural numbers.
Our target language is a simple machine assembly language:

\begin{align}
  \tag{Word}   n, p &\in \mathbb{N} \\
  \tag{Registers} r &::= ip \; | \; ep \; | \; r1 \; | \; r2  \\
  \tag{Write Operands}  wo &::= r \; | \; r\%n \\
  \tag{Read Operands}  ro &::= wo \; | \; n \\
  \tag{Basic Blocks} bb &::= i \; \# \; bb
                       | \; \texttt{jmp} \; \{(ro, n)\} \; ro \;  \\
  \tag{Instructions} i &::= \texttt{mov} \; ro \; wo \; 
                       | \; \texttt{new} \; n \; wo \; \\
  \notag   & \quad \;  | \; \texttt{push} \; ro \; 
                       | \; \texttt{pop} \; wo \\
\end{align}

Our machine words are natural numbers, which can be written into registers or
the heap, which is a finite map from pointers $p$ to words. Pointers can index
into the heap or into the program. Our machine consists Our instructions are a tiny subset of
standard instructions on modern machines. Ours consist of \texttt{mov}
instructions, direct and indirect \texttt{jmp}s with an optional conditional
location for when an argument is zero, , a \texttt{new} instruction that
returns a fresh heap location with \texttt{n} adjacent words, and \texttt{push}
and \texttt{pop}. Given our compiler, which we will describe in later sections,
which compiles a lambda term into a program, we prove the following main
result:

\begin{theorem}[Compiler Correctness]
If a term evaluates to a value, then the machine code executes to a state
representing that value. 
\end{theorem}

This is similar to the call-by-value results presented in
from~\cite{chlipala2007certified}. In the paper we will break down how the proof
of this theorem is broken down into separate modular phases, and combined
together to give the final proof. 

In addition to the above lemma, because call-by-need is an optimization of
call-by-name, we get the following corollary.

\begin{corollary}[Call-by-Name Correct Result]
If $t$ compiles to $p$, then $t$ evaluates to a value $v$ under call-by-name
semantics \emph{iff} $p$ executes on the instruction machine to $s$, where $s$
and $v$ are related appropriately.
\end{corollary}

This is a valuable corollary because it means a programmer or compiler writer
can reason about the simpler call-by-name semantics, and be confident that their
reasoning is preserved through compilation and execution.

\subsection{Contributions}
There are two primary contributions of this paper. 
\begin{itemize}
\item The first verified compiler of a higher order language that proves that
the machine semantics preserve time complexity of the source semantics
\item The first verified compiler of a non-strict language
\end{itemize}

\subsection{Outline}
The compiler and proofs use a number of different representations, including
standard lambda calculus, lambda calculus with deBruijn indices, machine
code, and small step and big step semantics. The paper proceeds by describing
each compilation step that the compiler makes, and sketching the proof that the
step preserves semantics.

Specifically, we start with background in Section \ref{sec:back}, then in
Section \ref{sec:cem} we cover the $\mathcal{CE}$ big step call-by-need
semantics. In Section \ref{sec:cesm} we describe the $\mathcal{CE}$ small step
semantics and how it relates to the big step semantics. In Section
\ref{sec:mach} we describe the instruction machine semantics and it's relation
to the $\mathcal{CE}$ small step semantics.  While this compiler is untyped, we
use Section \ref{sec:applications} to discuss how one could incorporate a type
system into the compiler. We then discuss threats to validity and future work in
Section \ref{sec:discussion} and conclude in Section \ref{sec:conclusion}. 

The compiler and all the proofs are available as Coq source code at
\texttt{https://github.com/stelleg/cem\_coq}.


